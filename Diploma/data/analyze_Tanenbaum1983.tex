\subsection{Практичный инструмент для создания портативных компиляторов} \label{subsection_Tanenbaum1983}
Статья\,\cite{Tanenbaum1983} описывает \todo{<<Амстердамский инструмент для компиляторов>>} который решает проблему написания \(N \times M\) программ. То есть для интерпритации и компиляции \(N\) языков программирования на \(M\) разных компьютерных архитектурах, требуется \(N \times M\) программ. <<Амстердамский инструмент для компиляторов>> решает эту проблему путем написания \(N\) программ, которые транслируют \(N\) языков в единое промежуточное представление и \(M\) программ, которые транслируют единое промежуточное представление в язык ассемблера для каждой из \(M\) архитектур. Таким образом нужно написать всего \(N + M\) программ. Для поддержания нового языка на всех архитектурах требуется написать всего 1 программу, также как и для поддержания всех языков новой архитектурой. 

Стоит отметить, что задача \(N \times M\) программ совсем не простая, и для ее решения авторам пришлось пойти на некоторые уступки, а именно, они ограничились только алгебраическими языками и 8-ми битными архитектурами.

\todo{Очень полезная статья. Похоже, что это <<дедушка>> всех современных инструментов, нужно изучить.}