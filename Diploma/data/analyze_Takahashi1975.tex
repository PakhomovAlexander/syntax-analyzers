\subsection{Обобщение регулярных множеств и их прикладные значения для изучения контекстно-свободных языков} \label{subsection_Takahashi1975}
Научная работа  \cite{Takahashi1975} Масакко Такаши проводит глубокий анализ регулярных множеств строк. В том числе деревьев и лесов, что бы это не значило. В первую очередь автор приводит ряд базовых определений и теорем (с доказательством) для регулярных выражений строк, и разбирает (опять же с математической точки зрения) их прикладное значение для языков с контекстно-свободным определением.

\todo{Я считаю, что это не то, что мне нужно, так как базовые определения даны в других источниках, а такие вещи, как Эс-локальные множества, наврятли мне понадобятся.} 