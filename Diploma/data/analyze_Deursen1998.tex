\subsection{Маленькие языки: Маленькая поддержка?} \label{subsection_Deursen1998}
В статье\,\cite{Deursen1998} приводится определение предметно-ориентированного языка и то, как он может помочь или навредить приложениям и бизнесу. 

Согластно статье\,\cite{Deursen1998}, предметно-ориентированный язык~--- небольшой, чаще декларативный, язык, выражающий различниые характеристики набора программ в конекретной проблеме конкретной предметной области. 

Предметно-ориентированное описание~--- программа (спецификация, описание, запрос, процесс, задача, ...), написанная на предметно-ориентированном языке.

Предметно-ориентированный процессор~--- програмное обеспечение для компиляции, интрепритации или анализа предметно-ориентированного языка.

Авторы приводят плюсы предметно-ориентрованных языков, например: легкость в написании предметных программ, простота эксплуатации и поддержиня таких програм, само-документирование. А также описываются и проблемы, такие, как сложность в расширении подобных язков, при необходимости расширения предметной области. 

В заключении, авторы говорят, что предметно-ориентированные языки это не панацея для решения всех прикладных задач, но хорошо спроектированные языки, с хорошим набором инструментов для работы с ними, могут значительно облегчить и ускорить процесс написания и поддержания программ.

Позитивный сценарий использования подобных языков: Вся реализация сложных процессов возложенна на компилятор, в то время как бизнес логика легко описывается с помощью языка. 

Негативный сценарий: Предметная область может быть не до конца изучена, для того что бы проектировать под нее специальный язык. В такой ситуации следует обратить внимание на стандартные средства написания программ. 

Альтернатива предметно-ориентированным языкам: Использование объекно-ориентированных языков вместе с библиотеками типов и функций, которые описывают предметную область. 

Возможные направления для улучшения предметно-ориентированных языков: Виуализация, приближение языков к естественным, симулирование среды для выполнения программ.

\todo{Полезная статья, из нее можно взять основные определения. Оставлю, если не найду что-то лучше.}