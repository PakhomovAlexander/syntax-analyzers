\subsection{Советы по написанию компиляторов и доказательтво их корректности} \label{subsection_Thatcher1981}
Статья\,\cite{Thatcher1981} описывает способ написания компилятора, который гарантированно является корректным с точки зрения математики. Авторы приводят 18 аргументов, которые доказывают абсолютную корректность алгоритма. <<Кооректность>> означает, что существует однозначная и связь исходного кода с машинным (скомпилированным), и она справедлива с точки зрения логики. 

В статье приведен мини-язык, на примере которого показывается корректность компилятора.

\todo{Много математических терминов, основа почти не дана. Статья является очень узкой (авторы описывают корректность компилятора, которы был озвучен на какой-то там конференции самим Моррисом(кто бы это не был)). Считаю, что дальше статью рассматривать не стоит.}