\subsection{Обзор методом интерпритации конекстно-свободных языков} \label{subsection_Lee1996}
В статья\,\cite{Lee1996} описываются основные подходы для интерпритации контекстно-свободных языков. В начале авторы описывают негативные стороны контекстно-свободных языков, а потом проводят верхнеуровневый обзор пяти методов интерпритации контекстно-свободных языков: 
\begin{enumerate} 
	\item{Методы, которые берут текст на вход.}
	\item{Методы, которые берут структурную информацию на вход.}
	\item{Методы, не основанные на контекстно-свободных языках.}
	\item{Методы, которые берут описание сабклассов.}
	\item{Стохастические методы.}
\end{enumerate}

В начале статьи описывается проблема изучения и интерпретации контекстно-свободных граматик. Затрагиваются темы интерпритации речи, изучения языка на лету, проблемы интерпритации граматик за полиномиальное время, и т.д.

\todo{Нужно изучить подробнее, так как описание самих методов довольно понятное и будет полезно для диплома.}