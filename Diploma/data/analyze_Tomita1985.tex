\subsection{Эффективный контекстно-свободный алгоритм для разбора естественных языков} \label{subsection_Tomita1985}
Статья\,\cite{Tomita1985} описывает алгоритм разбора естественных языков. Автор описывает существующие на то время алгоритмы разбора языков и объясняет почему они не годятся для разбора естественных языков.Он разделяет все алгоритмы на 2 группы:
\begin{enumerate} 
\item{Алгоритмы разбора языков программирования.}
\item{Алгоритмы разбора общих контексно свободных языков.}
\end{enumerate}

Обе группы, по словам автора, не подходят для разбора естественных языков. Первые~--- ограниченны слишком маленьким набором грамматик, вторые~--- слишком громоздкие для естественных языков, которые ближе к языкам программирования, а не к контекстно-свободным грамматикам.

Автор предлагает свой алгоритм, который находится между алгоримами разбора языков программирования и алгоритмами разбора контекстно-свободных языков. И подробно его описывает. 

\todo{Будет полезно для дипломной работы, стоит изучить статью.}