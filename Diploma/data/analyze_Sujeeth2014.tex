\subsection{Delite: Архитектура компилятора для встроенных производительных предметно-ориентированных языков} \label{subsection_Sujeeth2014}
Статья\,\cite{Sujeeth2014} описывает инструмент под названием Delite, разработанный Oracle Labs совместно со Стэндфорским Университетом, а так же предметно-ориентированные языки, написанные при помощи этого инструмента.

Delite позволяет упростить трудоёмкий процесс написания производительных предметно-ориентированных языков для работы с параллельным программированием. Простота написания подобных языков достигается за счет предоставления параллельных шаблонов проектирования, оптимизаций и кодо-генераций, которые направлены на использование в предметно-ориентированных языках программирования.

Инструмент интегрирован c языком Scala, а так же предоставляет возможность написания предметно-ориентированных языков, которые будут выполняться на C++, CUDA, OpenCL и MPI.

Зачем нужны подобные инструменты? Основная проблема заключается в том, что высокоуровневые языки программирования предоставляют понятные абстракции и читаемый код программ, но лишают пользователей использовать низкоуровневые оптимизации для достижения высокой производительности программ. Низкоуровневые языки, в свою очередь, предоставляют большие возможности по части оптимизации, но код программ становится очень сложным. 

Авторы статьи предлагают решить эту проблему с помощью предметно-ориентированных языков, которы позволят оптимизировать программы на низком уровни и в то же время сохранят читаемость кода.

Статья предлагает использовать Delite для написания подобных предметно-ориентированных языков, а так же описывает несколько уже существующих языков (OptiML для машинного обучения, OptiQL для запросов к данным, OptiGraph для анализа графов и OptiMesh для распределённых вычислений), разработанных с помощью Delite. 

Так же авторы статьи приводят анализ производительности, результаты которого сопоставимы с эквивалентными программ, написанными на C++.





\todo{}