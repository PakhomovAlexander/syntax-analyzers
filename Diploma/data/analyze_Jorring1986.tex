\subsection{Компиляторы и поэтапные преобразования} \label{subsection_Jorring1986}
Статья\,\cite{Jorring1986} содержит описание техник преобразования кода. А именно, \todo{<<предкомпиляция>>(precomputation)} и \todo{<<сокращение частотности>>(frequency reduction)}, обощенно их можно назвать <<поэтапные преобразования>>. Эти два приема автор демонстрирует на простых примерах, а также на фрагмете компиляции Пролога.

\todo{Мне кажется, это шаг в сторону инкрементальной компиляции. Наверно, было бы полезно разобраться в этом. Добавлю статью в список основных, если не найду тот же материал, но более доступным языком. Мне показалось, что статья немного непонятная для меня.}