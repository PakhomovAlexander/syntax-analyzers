\subsection{Как компиляторы должны сообщать об ошибках?} \label{subsection_Barik2018}
Статья\,\cite{Barik2018} описывает проблему непонятного объяснения ошибок компиляции и предлагает применить модель Тюльмина для повышения эффективности объяснения ошибок компиляции. Для полного понимания способа объяснения ошибок авторы статьи провели сравнительный анализ среди 68 разработчиков  и эмпирическое изучение сообщений об ошибках компиляции среди пользователей портала Stack Overflow.

Авторы выяснили, что короткое сообщение об ошибке, которое содержит решение проблемы, оказалось более предпочтительнее, чем развернутое сообщение об ошибке без предложений как её можно решить. 

Авторы утверждают, что если сообщение об ошибке оформлено как объяснение, к нему может быть применена теория объяснения Тюльмина, для того, что бы понять почему одни сообщения об ошибках эффективнее других.

В результате исследований, в статье предлагаются 3 принципа, которым нужно следовать при объяснении ошибок компиляции: 

\begin{enumerate} 
	\item{Предоставляйте разработчикам возможность самостоятельного выбора глубины объяснения проблемы}
	\item{Разделяйте понятия <<объяснения>> и <<решения>> проблемы}
	\item{Сопоставляйте сообщение об ошибке с содержимым выражения, где произошла ошибка}
\end{enumerate}


\todo{}