\subsection{Написание YACC на Sasl} \label{subsection_Jones1985}
Статья\,\cite{Jones1985} содержит описание процесса написания YACC на функциональном языке программирования Sasl. Автор приводит сравнение имперического подхода к программированию и функционального путем написания <<средней>> программы. 

В результате получиласть программа в 2 раза меньшая в объеме, чем такая же, но написанная на процедурном языке.

\todo{Очень интересно окунуться вглубь и посмотреть реализацию, но тема статьи не пересекается с темой дипломной работы. Разве что YACC, но совсем с другой стороны.}