\subsection{Семантико-направленный генератор компиляторов} \label{subsection_Paulson1982}
Статья\,\cite{Paulson1982} содержит описание генератора компиляторов, который Паульсон написал на Паскале. На вход он получает семантические грамматики, описанные в БНФ, а на выходе получается компилятор. Сам генератор состоит из трёх частей~--- это анализатор грамматик, универсальный транслятор и стэк машина. 

В качестве демонстрации работы генератора автор приводит описание самого языка Паскаль в БНФ и генерацию компилятора на её основе. В конце приводятся результаты сравнения скорости итоговых программ и их компиляций с помошью стандартного компилятора Паскаля и сгенерированного, где второй, очевидно, значимо проигрывает.

\todo{Интересная статья, но практического опыта из нее извлечь трудно, да и материал не новый. Думаю что не стоит ее разбирать глубже.}