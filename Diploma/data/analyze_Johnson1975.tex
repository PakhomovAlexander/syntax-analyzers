\subsection{Yacc, статья Стефана Джонсона} \label{subsection_Johnson1975}
Статья <<Еще один компилятор-компилятор>> \cite{Johnson1975} описывает Yacc. Yacc~--- это инструмент, предоставляющий пользователю возможность описать входящие структуры, а затем связать их с некоторыми действиями. По сути, Yacc является мощным инструментом разбора входящего потока строк, который потом предпринимает опредеоленные пользователем действия в случае совпадения некоторых правил, которы так же определены пользователем. 

Статья достаточно полно покрывает основные возможности Yacc, а так же приводит примеры их использования.

\todo{Yacc в любом случаю должен быть подробно разобран, статья хорошая, но пока оставлю ее на первом уровне, так как ожидаю найти более полный материал} 