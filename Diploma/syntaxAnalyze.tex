\subsection{Синтаксический анализатор} \label{sub113}

В~классической модели компилятора синтаксический анализатор получает строку токенов от~лексического анализатора и~проверяет, может ли~эта строка имен токенов порождаться грамматикой исходного языка. Также от~синтаксического анализатора ожидаются сообщения обо~всех выявленных ошибках и~умение продолжать работу с~оставшейся частью программы. В~случае корректной программы синтаксический анализатор строит дерево разбора и~передает его следующей части компилятора для дальнейшей обработки.

Имеется три основных типа синтаксических анализаторов грамматик: 

\begin{itemize} 
	\item{\textit{Универсальные методы разбора}, например, алгоритмы Кока-Янгера-Касами (Cocke-Younger-Kasami) и~Эрли (Earley)\cite{Earley1983} могут работать с~любой грамматикой. Однако эти обобщённые методы слишком неэффективны для использования в~промышленных компиляторах. }
	\item{\textit{Восходящие методы разбора} (bottom-up), построение дерева разбора происходит снизу (от~листьев) вверх (к~корню). Поток токенов сканируется слева направо. }
	\item{\textit{Нисходящие методы разбора} (top-down) строят дерево разбора сверху (от~корня) вниз (к~листьям). Входной поток синтаксического анализатора, как и~в~восходящих методах, сканируется посимвольно слева направо. }
\end{itemize}

\todo{Пример выражения на языке программирования и соответствующее ему синтаксическое дерево.}
