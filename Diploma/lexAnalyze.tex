\subsection{Лексический анализатор} \label{sub112}

При рассмотрении лексического анализа используются три термина:

\begin{itemize} 
	\item{
		\textit{Токен} (от~англ. \textit{token}~--- знак, символ) представляет собой пару, состоящую из~имени токена и~необязательного атрибута. Имя токена~--- абстрактный символ, представляющий тип лексической единицы, например, конкретное ключевое слово или последовательность символов, составляющих идентификатор. Имена токенов являются входными символами, обрабатываемыми синтаксическим анализатором. Атрибут токена~--- строка или структура, объединяющая несколько блоков информации. Блоки информации представляют собой описание лексемы (строчка кода, значение), которая представляет токен. 
		Выражение на~языке Fortran \texttt{E=M*2} будет представленно в~виде последовательности \\
		$\bnfpn{\textbf{id}, Указатель на запись в таблице символов для E}$ \\
		$\bnfpn{\textbf{assign\_op}}$ \\
		$\bnfpn{\textbf{id}, Указатель на запись в таблице символов для M}$ \\
		$\bnfpn{\textbf{mult\_op}}$ \\
		$\bnfpn{\textbf{number}, Целое значение 2}$ \\
		Примеры токенов приведены в~табл.~\ref{tokens}.
		\begin{table} [h!tbp]
			\centering
			\changecaptionwidth\captionwidth{15.35cm}
			\caption{Примеры токенов}\label{tokens}%
			\begin{tabular}{| p{3cm} | p{6cm} | p{5cm} |} \hline
				\textbf{Токен}		&	\textbf{Неформальное описание}												&	\textbf{Примеры лексем}						\\ \hline
				\textbf{if}  		& 	Символы \texttt{i}, \texttt{f} 												& 	\texttt{if} 								\\ \hline
				\textbf{else}  		& 	Символы \texttt{e}, \texttt{l}, \texttt{s} \texttt{e} 						& 	\texttt{else} 								\\ \hline
				\textbf{comparison}	& 	\texttt{<}, \texttt{>}, \texttt{<=}, \texttt{>=}, \texttt{==}, \texttt{!=}	& 	\texttt{<=}, \texttt{!=} 					\\ \hline
				\textbf{id}  		& 	Буква, за~которой следуют буквы и~цифры										& 	\texttt{pi}, \texttt{score}, \texttt{D2}	\\ \hline
				\textbf{number}  	& 	Любая числовая константа													& 	\texttt{3.14159}, \texttt{0} 				\\ \hline
				\textbf{literal}  	& 	Все символы, заключенные в~двойные кавычки, кроме самих кавычек				& 	\texttt{"core dumped"} 						\\ \hline	
			\end{tabular}
		\end{table}	
	}
	\item{\textit{Шаблон}~--- это описание вида, который может принимать лексема токена. В~случае ключевого слова шаблон представляет собой последовательность символов, образующую это ключевое слово. Для некоторых токенов шаблон представляет более сложную структуру (например, регулярное выражение)~\cite{web2}.}
	\item{\textit{Лексема}~--- последовательность символов исходной программы, которая соответствует шаблону токена и~идентифицируется лексическим анализатором как экземпляр токена.}
\end{itemize}

Основная задача лексического анализатора~--- чтение входных символов исходной программы, группировка их ~в~лексемы и~вывод последовательностей токенов для всех лексем исходной программы. Поток токенов пересылается синтаксическому анализатору для разбора. Лексический анализатор удаляет комментарии, пробельные символы, синхронизирует сообщения об~ошибках и раскрывает макросы.