\subsection{Лексический анализ} \label{sub111}

Основная задача лексического анализатора состоит в чтении входных символов исходной программы, их группировании в лексемы и вывод последовательностей токенов для всех лексем исходной программы. Поток токенов пересылается синтаксическому анализатору для разбора. Так же лексический анализатор может удалять комментарии, пробельные символы, синхронизировать сообщения об ошибках или раскрывать макросы.

Фаза анализа компиляции разделена на лексический и синтаксический анализ по ряду причин: 

\begin{enumerate} 
	\item{Упрощение разработки. Отделение лексического анализа от синтаксического позволяет упростить как минимум одну из фаз анализа. Например, включить в синтаксический анализатор работу с комментариями и пробельными символами существенно сложнее, чем удалить их лексическим анализатором.}
	\item{Увеличение эффективности компилятора. Отдельный лексический анализатор позволяет применять более специализированные методики, предназначенные исключительно для решения лексических задач.}
	\item{Увеличение переносимости компилятора. Особенности устройств ввода могут быть ограниченны лексическим анализатором.}
\end{enumerate}


При рассмотрении лексического анализа используются три связанных, но различных термина:


\begin{itemize} 
	\item{\textit{Токен} представляет собой пару, состоящую из имени токена и необязательного атрибута. Имя токена~--- абстрактный символ, представляющий тип лексической единицы, например конкретное ключевое слово или последовательность символов, составляющих идентификатор. Имена токенов являются входными символами, обрабатываемыми синтаксическим анализатором.}
	\item{\textit{Шаблон}~--- это описание вида, который может принимать лексема токена. В случае ключевого слова шаблон представляет собой последовательность символов, образующая это ключевое слово. Для некоторых токенов шаблон представляет более сложную структуру.}
	\item{\textit{Лексема}~--- последовательность символов исходной программы, которая соответствует шаблону токена и идентифицируется лексическим анализатором как экзепляр токена.}
	\item{Атрибут токена~--- строка или структура, объединяющая несколько блоков информации, которая содержит значение числа, в случае если токен соответствует шаблону числа, или описание лексемы (строчка кода, значение) которая представляет токен. \todo{Пример?}}
\end{itemize}


\todo{Таблица примеров токена?}

