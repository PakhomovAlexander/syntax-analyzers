\section{Постановка задачи} \label{sub21}

Целью работы является обеспечение поддержки (подсветки синтаксиса и~авто-дополнения исходного кода) предметно-ориентированного языка программирования в~интерактивной среде разработки (IDE) IntelliJ~IDEA.

Задачи работы: 

\begin{enumerate} 
\item{Изучить существующие подходы к~лексическому и~синтаксическому анализу языков программирования с~контекстно-свободной грамматикой и~динамической компиляцией.}	
\item{Разработать лексический и~синтаксический анализаторы для подсветки синтаксиса и~авто-дополнения исходного кода для предоставленного языка (предметно ориентированного) GRIP~DSL.}
\item{Интегрировать лексический и~синтаксический анализаторы с~Intellij~IDEA.}
\end{enumerate} 

Областью исследования является лексический и~синтаксический анализ, а~также интеграция этих процессов с~современной средой разработки.

Предмет исследования~--- подцветка синтаксиса Groovy~DSL в~Intellij~IDEA.

Разработка полноценного плагина для IDE, самостоятельного компилятора и~предметно-ориентированного языка выходят за~рамки работы.
