\section{Постановка задачи} \label{sub21}

Целью работы является обеспечение поддержки (подсветки синтаксиса и~автодополнения исходного кода) предметно-ориентированного языка программирования GRIP~DSL в~интерактивной среде разработки (IDE) IntelliJ~IDEA.

Задачи работы: 

\begin{enumerate} 
\item{Изучить существующие подходы к~лексическому и~синтаксическому анализу языков программирования с~контекстно-свободной грамматикой и~динамической компиляцией.}	
\item{Разработать GDSL-модуль для подсветки синтаксиса \\ и~автодополнения исходного кода для языка GRIP~DSL в~интерактивной среде разработки Intellij~IDEA.}
\end{enumerate} 

Областью исследования является лексический и~синтаксический анализ, а~также интеграция этих процессов с~современной средой разработки.

Предметом исследования является подсветка и автодополнение синтаксиса Groovy~DSL в~Intellij~IDEA. 

Работа не~включает в~себя разработка полноценного плагина для IDE, самостоятельного компилятора и~предметно-ориентированного языка.
