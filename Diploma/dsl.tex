\subsection{Определение предметно-ориентированного языка программирования} \label{sub121}

\textit{Предметно-ориентированный язык программирования (domain-specific language)}~--- язык программирования или исполняемая спецификация, которые предлагают выразительное и~мощное решение конкретной предметной проблемы с~помощью высокоуровневых абстракций и~специализированных нотаций. 

Чаще всего такой язык программирования не~является большим и~предоставляет только те~конструкции, которые необходимы для решения предметной задачи (например, язык Yacc). Но~иногда такие языки являются подмножеством других языков программирования широкого применения. Такой подход позволяет совместить выразительность и~мощь предметно-ориентированного языка вместе с~возможностями языков широкого применения. Такие языки называются \textit{встраиваемыми (embedded languages)}~\cite{VanDeursen2000}.

Альтернативой предметно-ориентированному языку является использование объектно-ориентированного языка программирования вместе с~библиотекой типов и~функций, которые отвечают предметным потребностям.~\cite{Deursen1998}