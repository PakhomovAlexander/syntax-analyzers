\subsection{Определение предметно-ориентированного языка программирования} \label{sub121}

\textit{Предметно-ориентированный язык программирования (domain-specific language, DSL)}~--- язык программирования или исполняемая спецификация, которые предлагают выразительное и~мощное решение конкретной предметной проблемы с~помощью высокоуровневых абстракций и~специализированных нотаций. Чаще всего такой язык программирования не~является большим и~предоставляет только те~конструкции, которые необходимы для решения предметной задачи (например, язык Yacc).

Альтернативой предметно-ориентированному языку является использование объектно-ориентированного языка программирования вместе с~библиотекой типов и~функций, которые отвечают предметным потребностям.~\cite{Deursen1998}