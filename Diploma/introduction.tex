\chapter*{Введение}                         % Заголовок
\addcontentsline{toc}{chapter}{Введение}    % Добавляем его в оглавление

%\newcommand{\actuality}{}
%\newcommand{\progress}{}
%\newcommand{\aim}{{\textbf\aimTXT}}
%\newcommand{\tasks}{\textbf{\tasksTXT}}
%\newcommand{\novelty}{\textbf{\noveltyTXT}}
%\newcommand{\influence}{\textbf{\influenceTXT}}
%\newcommand{\methods}{\textbf{\methodsTXT}}
%\newcommand{\defpositions}{\textbf{\defpositionsTXT}}
%\newcommand{\reliability}{\textbf{\reliabilityTXT}}
%\newcommand{\probation}{\textbf{\probationTXT}}
%\newcommand{\contribution}{\textbf{\contributionTXT}}
%\newcommand{\publications}{\textbf{\publicationsTXT}}


Современная разработка програмного обеспечения стремительно движется в~сторону более открытого и~гибкого взаимодействия с~пользователем приложения.
Достигается это по-разному. 

Одним из~способов взаимодействия с~пользователем является графический интерфейс. Такой подход позволяет скрыть детали работы приложения и~предоставить возможность взаимодействовать с~системой только с~помошью графических элементов управления.

Другим инструментом взаимодействия пользователя с приложением являются предметно-ориентированные языки программирования (DSL). Основным отличием DSL от~графического интерфейса является гибкость, которую DSL предоставляет пользователю приложения. С~помощью DSL можно описать бесконечное количество сценариев взаимодействия с~системой. 

Однако, у~такого подхода существуют и~минусы. Например, для того, что~бы~пользователь начал писать код на~DSL, ему требуется время на~изучение синтаксиса языка. Второй проблемой является выбор среды разработки для DSL, которая позволяла бы набирать код на~DSL так~же~удобно, как и~на~языках общего применения. 

Компания Gridfore активно разрабатывает собственный DSL (GRIP DSL) для удобного управления потоками данных. Главной задачей Gridfore является разработать максимально простой и~поддерживаемый DSL, с~которым могли бы~взаимодействовать пользователи, не~имеющие опыта разработки на~языках программирования общего применения. Одной из~приоритетных задач является обеспечение поддержки GRIP~DSL в~интегрированной среде разработки IntelliJ IDEA.  

%\textbf{Объем и структура работы.}
%% на случай ошибок оставляю исходный кусок на месте, закомментированным
%Полный объём диссертации составляет  \ref*{TotPages}~страницу
%с~\totalfigures{}~рисунками и~\totaltables{}~таблицами. Список литературы
%содержит \total{citenum}~наименований.
%
%Полный объём дипломной работы составляет
%\formbytotal{TotPages}{страниц}{у}{ы}{}, включая
%\formbytotal{totalcount@figure}{рисун}{ок}{ка}{ков} и
%\formbytotal{totalcount@table}{таблиц}{у}{ы}{}.   Список литературы содержит
%\formbytotal{citenum}{наименован}{ие}{ия}{ий}.
