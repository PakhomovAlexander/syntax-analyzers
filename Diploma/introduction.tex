\chapter*{Введение}                         % Заголовок
\addcontentsline{toc}{chapter}{Введение}    % Добавляем его в оглавление

%\newcommand{\actuality}{}
%\newcommand{\progress}{}
%\newcommand{\aim}{{\textbf\aimTXT}}
%\newcommand{\tasks}{\textbf{\tasksTXT}}
%\newcommand{\novelty}{\textbf{\noveltyTXT}}
%\newcommand{\influence}{\textbf{\influenceTXT}}
%\newcommand{\methods}{\textbf{\methodsTXT}}
%\newcommand{\defpositions}{\textbf{\defpositionsTXT}}
%\newcommand{\reliability}{\textbf{\reliabilityTXT}}
%\newcommand{\probation}{\textbf{\probationTXT}}
%\newcommand{\contribution}{\textbf{\contributionTXT}}
%\newcommand{\publications}{\textbf{\publicationsTXT}}


Современная разработка програмного обеспечения стремительно движется в~сторону более открытого и~гибкого взаимодействия с~пользователем (программист, аналитик и~т.д.) системы. 

Одним из~способов взаимодействия с~пользователем является графический интерфейс. Такой подход позволяет скрыть детали работы приложения и~предоставить возможность взаимодействовать с~системой только с~помошью графических элементов управления.

Другим инструментом взаимодействия пользователя с~приложением являются предметно-ориентированные языки программирования (англ. Domain Specific Language, DSL). Основным отличием языка DSL от~графического интерфейса является гибкость, которую DSL предоставляет пользователю приложения. С~помощью DSL можно описать сценарии взаимодействия с~системой. 

Однако, у~такого подхода существуют и~минусы. Например, для того, чтобы~пользователь начал писать код на~DSL, ему требуется время на~изучение синтаксиса языка. Одним из решений является выбор среды разработки, которая позволяла бы~писать код на~языке DSL так~же~удобно, как и~на~языках общего применения. 

В данной работе рассматривается язык GRIP~DSL для предметной области класса DMP (Data Management Platform) c~возможностью аналитики в~реальном времени, разработкой которой занимается компания Gridfore. Язык GRIP~DSL предназначен для разработки сценариев управления потоками данных, с~которым могли бы~взаимодействовать пользователи, не~имеющие опыта разработки на~языках программирования общего применения.

Целью работы является обеспечение поддержки языка программирования GRIP~DSL в~интегрированной среде разработки IntelliJ~IDEA.

Задачи работы: 
\begin{enumerate} 
	\item{Изучить существующие подходы к~лексическому и~синтаксическому анализу языков программирования с~контекстно-свободной грамматикой и~динамической компиляцией.}	
	\item{Разработать лексический и синтаксический анализаторы с~целью подсветки синтаксиса и~автодополнения исходного кода для языка GRIP~DSL.}
\end{enumerate} 


%\textbf{Объем и структура работы.}
%% на случай ошибок оставляю исходный кусок на месте, закомментированным
%Полный объём диссертации составляет  \ref*{TotPages}~страницу
%с~\totalfigures{}~рисунками и~\totaltables{}~таблицами. Список литературы
%содержит \total{citenum}~наименований.
%
%Полный объём дипломной работы составляет
%\formbytotal{TotPages}{страниц}{у}{ы}{}, включая
%\formbytotal{totalcount@figure}{рисун}{ок}{ка}{ков} и
%\formbytotal{totalcount@table}{таблиц}{у}{ы}{}.   Список литературы содержит
%\formbytotal{citenum}{наименован}{ие}{ия}{ий}.
