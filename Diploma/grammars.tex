\subsection{Формальное определение контекстно-свободной грамматики} \label{sub114}

Для работы синтаксического анализатора требуется описание грамматики языка программирования, которое задаётся с помощью контекстно-свободных грамматик (далее КС-грамматики).

Существует множество типов грамматик (грамматики типа~3, контекстно-свободные грамматики, контекстно-зависимые грамматики и грамматики без ограничений), с~помощью которых можно описать различные языки. Для описания языков программирования используются КС-грамматики, потому что их разбор наиболее быстрый.

КС-грамматика используется для определения формального синтаксиса языка программирования. КС-грамматика естественным образом описывает иерархическую структуру множества конструкций языка программирования. Ниже приведён пример КС-грамматики, которая описывает выражение \texttt{switch} языка \texttt{Java}:
\begin{align*}
SwitchStatement &\to \textbf{switch}\;\textbf{(}\; Expression\; \textbf{)}\; SwitchBlock \\
SwitchBlock &\to \textbf{\{}\; \textbf{\{}\;SwitchBlockStatements\;\textbf{\}}\; \textbf{\{}\;SwitchLabel\;\textbf{\}}\; \textbf{\}} \\
SwitchBlockStatements &\to SwitchLabels\; BlockStatements \\
SwitchLabels &\to SwitchLabel\; \textbf{\{}\;SwitchLabel\;\textbf{\}} \\
SwitchLabel &\to \textbf{case}\; ConstantExpression\; \textbf{:} \\
SwitchLabel &\to \textbf{case}\; EnumConstantName \;\textbf{:} \\
SwitchLabel &\to \textbf{default\;:} \\
EnumConstantName &\to Identifier \\
ConstantExpression &\to Expression
\end{align*}

КС-грамматика состоит из~четырёх частей:
 
\begin{enumerate} 
	\item{\textit{Терминалы}~--- базовые символы, формирующие строки. Термин <<Имя токена>> является синонимом слова <<Терминал>>. Пример терминала: \texttt{switch}, \texttt{case}, \texttt{(}, \texttt{)}}. Другими словами, это листья дерева разбора, представленного на~рис.~\ref{img:tree}.
	\item{\textit{Нетерминалы}~--- синтаксические переменные, которые обозначают множества строк. В~примере, приведённом выше, $\begin{aligned} Expression \end{aligned}$, $\begin{aligned} SwitchBlock \end{aligned}$, $\begin{aligned} SwitchLabel \end{aligned}$ и~др. являются нетерминалами. Эти множества строк, обозначаемые нетерминалами, помогают определить язык, порождаемый КС-грамматикой. Нетерминалы также налагают на~язык иерархическую структуру, облегчающую синтаксический анализ и~трансляцию.}
	\item{\textit{Стартовый символ}~--- один из~нетерминалов, который обозначает множество строк, определяемых КС-грамматикой. По~соглашению, продукции стартового символа указываются первыми (в примере это $\begin{aligned} SwitchStatement \end{aligned}$).}
	\item{\textit{Продукция}~--- способ, которым терминалы и~нетерминалы объединяются в~строки. Каждая продукция состоит из~заголовка (левая часть), символа $\begin{aligned} &\to \end{aligned}$ и~тела (правая часть), которое состоит из~нуля или некоторого количества терминалов и~нетерминалов.}
\end{enumerate}
