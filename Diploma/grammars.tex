\subsection{Формальное определение контекстно-свободной грамматики} \label{sub114}

Существует множество грамматик (грамматики типа 3, контекстно-свободные грамматики, контекстно-зависимые грамматики и грамматики без ограничений), с помощью которых можно описать формальные языки. Для описания языков программирования используются контекстно-свободные грамматики  (далее КС-грамматики).

КС-грамматика используется для определения синтаксиса языка программирования. КС-грамматика естественным образом описывает иерархическую структуру множества конструкций языка программирования. Состоит она из четырёх частей:

\begin{enumerate} 
	\item{\textit{Терминалы}~--- базовые символы, формирующие строки. Термин <<имя токена>> является синонимом слова <<терминал>>. Пример терминала: \texttt{if}, \texttt{else}, \texttt{(}, \texttt{)}}.
	\item{\textit{Нетерминалы}~--- синтаксические переменные, которые обозначают множества строк. В примере $\begin{aligned} statement \end{aligned}$ и $\begin{aligned} expression \end{aligned}$ являются нетерминалами. Эти множества строк, обозначаемые нетерминалами, помогают определить язык, порождаемый КС-грамматикой. Нетерминалы также налагают на язык иерархическую структуру, облегчающую синтаксический анализ и трансляцию.}
	\item{\textit{Стартовый символ}~--- один из нетерминалов, который обозначает множество строк, являющиеся языком, определяемым КС-грамматикой. По соглашению, продукции стартового символа указываются первыми.}
	\item{\textit{Продукция}~--- способ, которым терминалы и нетерминалы объединяются в строки. Каждая продукция состоит из заголовка(левая часть), символа $\begin{aligned} &\to \end{aligned}$ и тела (правая часть), которое состоит из нуля или некоторого количества терминалов и нетерминалов.}
\end{enumerate}
\begin{align*}
expression &\to expression + term \\
expression &\to expression - term \\
expression &\to term \\
term &\to term * factor \\
term &\to term / factor \\ 
term &\to factor \\
factor &\to (expression) \\
factor &\to \textbf{id}
\end{align*}