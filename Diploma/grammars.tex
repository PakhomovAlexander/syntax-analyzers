\subsection{Формальное определение контекстно-свободной грамматики} \label{sub114}

Контекстно-свободная грамматика (далее грамматика) используется для определения синтаксиса языка программирования. Грамматика естественным образом описывает иерархическую структуру множества конструкций языка программирования. Состоит она из четырёх частей:

\begin{enumerate} 
	\item{\textit{Терминалы}~--- базовые символы, формируются строки. Термин <<имя токена>> является синонимом слова <<терминал>>. Пример терминала: $\bnfts{if}$, $\bnfts{else}$, $\bnfts{(}$, $\bnfts{)}$}
	\item{\textit{Нетерминалы}~--- синтаксические переменные, которые обозначают множества строк. В примере $\bnfpn{statement}$ и $\bnfpn{expression}$ являются нетерминалами. Эти множества строк, обозначаемые нетерминалами, помогают определить язык, порождаемый грамматикой. Нетерминалы также налагают на язык иерархическую структуру, облегчающую синтаксический анализ и трансляцию.}
	\item{\textit{Стартовый символ}~--- один из нетерминалов, который обозначает множество строк, являющиеся языком, определяемым грамматикой. По соглашению, продукции стартового символа указываются первыми.}
	\item{\textit{Продукция}~--- способ, которым терминалы и нетерминалы объединяются в строки. Каждая продукция состоит из заголовка(левая часть), символа $\bnfpo$ и тела (правая часть), которое состоит из нуля или некоторого количества терминалов и нетерминалов.}
\end{enumerate}


\begin{bnf*}
	\bnfprod{expression}
	{\bnfpn{expression} \bnfsp \bnfts{+} \bnfsp \bnfpn{term}}\\
	\bnfprod{expression}
	{\bnfpn{expression} \bnfsp \bnfts{-} \bnfsp \bnfpn{term}}\\
	\bnfprod{expression}
	{\bnfpn{term}}\\
	\bnfprod{term}
	{\bnfpn{term} \bnfsp \bnfts{*} \bnfsp \bnfpn{factor}}\\
	\bnfprod{term}
	{\bnfpn{term} \bnfsp \bnfts{/} \bnfsp \bnfpn{factor}}\\
	\bnfprod{term}
	{\bnfpn{factor}}\\
	\bnfprod{factor}
	{\bnfts{(} \bnfsp \bnfpn{expression} \bnfts{)}}\\
\end{bnf*}