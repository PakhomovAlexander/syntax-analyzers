\chapter*{Заключение}                       % Заголовок
\addcontentsline{toc}{chapter}{Заключение}  % Добавляем его в оглавление

%% Согласно ГОСТ Р 7.0.11-2011:
%% 5.3.3 В заключении диссертации излагают итоги выполненного исследования, рекомендации, перспективы дальнейшей разработки темы.
%% 9.2.3 В заключении автореферата диссертации излагают итоги данного исследования, рекомендации и перспективы дальнейшей разработки темы.
%% Поэтому имеет смысл сделать эту часть общей и загрузить из одного файла в автореферат и в диссертацию:

В~результате проделанной работы 

\begin{itemize} 
	\item{изучены существующие подходы к~лексическому и~синтаксическому анализу языков программирования с~контекстно-свободной грамматикой и~динамической компиляцией;}	
	\item{разработан GDSL-модуль для подсветки синтаксиса и~авто-дополнения исходного кода для языка GRIP~DSL.}
\end{itemize} 

В результате чего обеспечена подсветка синтаксиса и~авто-дополнения исходного кода языка GRIP~DSL в~интегрированной среде разработки IntelliJ~IDEA.
