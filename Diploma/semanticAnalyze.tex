\subsection{Семантический анализатор} \label{sub115}

Семантический анализатор использует синтаксическое дерево и~информацию из~таблицы символов для проверки исходной программы на~семантическую согласованность с~определением языка. Он также собирает информацию о~типах и~сохраняет ее в~синтаксическом дереве или в~таблице символов для последующего использования в процессе генерации промежуточного кода.

Важной частью семантического анализа является проверка типов, когда компилятор проверяет, имеет ли~каждый оператор операнды соответствующего типа. Например, многие определения языков программирования требуют, чтобы индекс массива был целым неотрицательным числом. Компилятор должен сообщить об ошибке, если в~качестве индекса массива используется число с~плавающей точкой или целое отрицательное число~\cite{web1}.