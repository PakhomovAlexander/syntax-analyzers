\subsection{Виды предметно-ориентированных языков программирования} \label{sub122}
Глобально предметно-ориентированные языки программирования делятся на~2~группы: 
\begin{itemize} 
	\item{\textit{внешние}}
	\item{\textit{внутренние}}
\end{itemize}

Разработка внешних языков состоит из~трёх шагов:

\begin{itemize} 
	\item{определение семантической модели}
	\item{определение синтаксической модели (абстрактный и~конкретный синтаксис)}
	\item{определение правил трансформации (правила, по~которым абстрактное представление транслируется в~исполнимое}
\end{itemize}

Для генерации
лексического и~синтаксического анализатора внешних языков существуют готовые средства, например, связка программ Lex~+~Yacc, входящая в~стандарт POSIX. 

Плюсом внешних DSL является узкая специализация, что облегчает процесс решения предметных задач, а~так~же гибкая базовая грамматика. Но~у~внешних языков существует и~ряд недостатков. Например, среду разработки, которая поддерживала и~облегчала бы написание сценариев на~внешнем
DSL, обычно разрабатывают либо с~нуля, либо как
дополнение к~уже существующей современной интегрированной среде разработки (integrated
development environment, IDE). Также, наряду с~лёгкостью решения предметных задач, внешние DSL практически никогда не~подходят для решения задач в~смежных областях.

Альтернативой внешним DSL являются внутренние. Внутренний DSL (embedded language)~--- это подмножество других языков программирования широкого применения~\cite{VanDeursen2000}. Такой подход позволяет совместить выразительность и~мощь предметно-ориентированного языка вместе с~возможностями языков широкого применения (Groovy, Scala, Kotlin, Ruby, Python, C\#, F\#, Haskell). 

Выбирая современный язык программирования общего назначения как основу для создания
внутреннего DSL, мы сразу получаем готовый
набор средств поддержки разработки~--- современные IDE, которые поддерживают базовый язык~\cite{Botov}.
Но не~во~всех случаях такая поддержка является полноценной \todo{какая-то подводочка к проблеме плохой подцветки Groovy}.

\todo{Схема внутренних и внешних DSL}

