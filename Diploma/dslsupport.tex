\subsection{Поддержка предметно-ориентированных языков программирования в IntelliJ IDE} \label{sub132}

Поддержка языка~--- процесс лексического, синтаксического и~семантического анализа c целью навигации, автодополнения и подсветки синтаксиса исходного кода, написанного на поддерживаемом языке программирования. 

Для реализации поддержки различных языков программирования платформа IntelliJ предоставляет возможность добавления собственных лексических и синтаксических анализаторов с последующей интеграцией с~IntelliJ.

Платформа IntelliJ преобразует исходный код в \textit{PSI(Program Structure Interface)}. PSI~--- набор функциональных возможностей, который предназначен для анализа файлов, построения синтаксических моделей исходного кода и создания индексов. Функциональные возможности PSI включают в себя:

\begin{itemize}
\item{быструю навигацию по файлам;}
\item{подсветку синтаксиса;}
\item{динамические проверки кода;}
\item{исправление кода, включая обширный рефакторинг;}
\item{автодополнение кода и д.р.}
\end{itemize}

Поддержка языка DSL, в~частности внутреннего, отличается от стандартной поддержки языка программирования общего назначения. Отличие заключается в том, что лексический и синтаксический анализ уже проведены для расширяемого языка программирования. Разработка модуля поддержи языка DSL сводится к правильному использованию уже готовых компонентов (PSI).
