%%% Проверка используемого TeX-движка %%%
\RequirePackage{ifxetex, ifluatex}
\newif\ifxetexorluatex   % определяем новый условный оператор (http://tex.stackexchange.com/a/47579)
\ifxetex
    \xetexorluatextrue
\else
    \ifluatex
        \xetexorluatextrue
    \else
        \xetexorluatexfalse
    \fi
\fi

\newif\ifsynopsis           % Условие, проверяющее, что документ --- автореферат

\RequirePackage{etoolbox}[2015/08/02]               % Для продвинутой проверки разных условий

%%% Поля и разметка страницы %%%
\usepackage{pdflscape}                              % Для включения альбомных страниц
\usepackage{geometry}                               % Для последующего задания полей

%%% БНФ %%%
\usepackage[epsilon]{backnaur}

%%% Установка indent в оглавлении %%%
\usepackage{tocloft}

%%% Математические пакеты %%%
\usepackage{amsthm,amsmath,amscd}   % Математические дополнения от AMS
\usepackage{amsfonts,amssymb}       % Математические дополнения от AMS
\usepackage{mathtools}              % Добавляет окружение multlined

%%%% Установки для размера шрифта 14 pt %%%%
%% Формирование переменных и констант для сравнения (один раз для всех подключаемых файлов)%%
%% должно располагаться до вызова пакета fontspec или polyglossia, потому что они сбивают его работу
\newlength{\curtextsize}
\newlength{\bigtextsize}
\setlength{\bigtextsize}{13.9pt}

\makeatletter
%\show\f@size                                       % неплохо для отслеживания, но вызывает стопорение процесса, если документ компилируется без команды  -interaction=nonstopmode 
\setlength{\curtextsize}{\f@size pt}
\makeatother

%%% Кодировки и шрифты %%%
\ifxetexorluatex
    \usepackage{polyglossia}[2014/05/21]            % Поддержка многоязычности (fontspec подгружается автоматически)
\else
   %%% Решение проблемы копирования текста в буфер кракозябрами
    \ifnumequal{\value{usealtfont}}{0}{}{
        \input glyphtounicode.tex
        \input glyphtounicode-cmr.tex %from pdfx package
        \pdfgentounicode=1
    }
    \usepackage{cmap}                               % Улучшенный поиск русских слов в полученном pdf-файле
    \ifnumequal{\value{usealtfont}}{2}{}{
        \defaulthyphenchar=127                      % Если стоит до fontenc, то переносы не впишутся в выделяемый текст при копировании его в буфер обмена
    }
    \usepackage{textcomp}
    \usepackage[T1,T2A]{fontenc}                    % Поддержка русских букв
    \ifnumequal{\value{usealtfont}}{1}{% Используется pscyr, при наличии
        \IfFileExists{pscyr.sty}{\usepackage{pscyr}}{}  % Подключение pscyr
    }{}
    \usepackage[utf8]{inputenc}[2014/04/30]         % Кодировка utf8
    \usepackage[english, russian]{babel}[2014/03/24]% Языки: русский, английский
    \ifnumequal{\value{usealtfont}}{2}{
        % http://dxdy.ru/post1238763.html#p1238763
        \usepackage[scaled=0.960]{XCharter}[2017/12/19] % Подключение русифицированных шрифтов XCharter
        \usepackage[charter, vvarbb, scaled=1.048]{newtxmath}[2017/12/14]
        \setDisplayskipStretch{-0.078}
    }{}
\fi

%%% Оформление абзацев %%%
\usepackage{indentfirst}                            % Красная строка

%%% Цвета %%%
\usepackage[dvipsnames, table, hyperref, cmyk]{xcolor} % Совместимо с tikz. Конвертация всех цветов в cmyk заложена как удовлетворение возможного требования типографий. Возможно конвертирование и в rgb.

%%% Таблицы %%%
\usepackage{longtable,ltcaption}                    % Длинные таблицы
\usepackage{multirow,makecell}                      % Улучшенное форматирование таблиц

%%% Общее форматирование
\usepackage{soulutf8}                               % Поддержка переносоустойчивых подчёркиваний и зачёркиваний
\usepackage{icomma}                                 % Запятая в десятичных дробях

%%% Оптимизация расстановки переносов и длины последней строки абзаца
\ifluatex
    \ifnumequal{\value{draft}}{1}{% Черновик
        \usepackage[hyphenation, lastparline, nosingleletter, homeoarchy,
        rivers, draft]{impnattypo}
    }{% Чистовик
        \usepackage[hyphenation, lastparline, nosingleletter]{impnattypo}
    }
\else
    \usepackage[hyphenation, lastparline]{impnattypo}
\fi

%%% Гиперссылки %%%
\usepackage{hyperref}[2012/11/06]

%%% Изображения %%%
\usepackage{graphicx}[2014/04/25]                   % Подключаем пакет работы с графикой

%%% Списки %%%
\usepackage{enumitem}

%%% Счётчики %%%
\usepackage[figure,table]{totalcount}               % Счётчик рисунков и таблиц
\usepackage{totcount}                               % Пакет создания счётчиков на основе последнего номера подсчитываемого элемента (может требовать дважды компилировать документ)
\usepackage{totpages}                               % Счётчик страниц, совместимый с hyperref (ссылается на номер последней страницы). Желательно ставить последним пакетом в преамбуле

%%% Продвинутое управление групповыми ссылками (пока только формулами) %%%
\ifxetexorluatex
    \usepackage{cleveref}                           % cleveref корректно считывает язык из настроек polyglossia
\else
    \usepackage[russian]{cleveref}                  % cleveref имеет сложности со считыванием языка из babel. Такое решение русификации вывода выбрано вместо определения в documentclass из опасности что-то лишнее передать во все остальные пакеты, включая библиографию.
\fi
\creflabelformat{equation}{#2#1#3}                  % Формат по умолчанию ставил круглые скобки вокруг каждого номера ссылки, теперь просто номера ссылок без какого-либо дополнительного оформления
\crefrangelabelformat{equation}{#3#1#4\cyrdash#5#2#6}   % Интервалы в русском языке принято делать через тире, если иное не оговорено


\ifnumequal{\value{draft}}{1}{% Черновик
    \usepackage[firstpage]{draftwatermark}
    \SetWatermarkText{DRAFT}
    \SetWatermarkFontSize{14pt}
    \SetWatermarkScale{15}
    \SetWatermarkAngle{45}
}{}

%%% Цитата, не приводимая в автореферате:
% возможно, актуальна только для biblatex
%\newcommand{\citeinsynopsis}[1]{\ifsynopsis\else ~\cite{#1} \fi}