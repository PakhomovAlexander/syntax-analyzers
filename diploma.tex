%&preformat-disser
\RequirePackage[l2tabu,orthodox]{nag} % Раскомментировав, можно в логе получать рекомендации относительно правильного использования пакетов и предупреждения об устаревших и нерекомендуемых пакетах
% Формат А4, 14pt (ГОСТ Р 7.0.11-2011, 5.3.6)
\documentclass[a4paper,14pt,oneside,openany]{memoir}

\input{common/setup}            % общие настройки шаблона
\input{common/packages}         % Пакеты общие для диссертации и автореферата
\synopsisfalse                      % Этот документ --- не автореферат
\input{Diploma/dispackages}    % Пакеты для диссертации
\input{Diploma/userpackages}   % Пакеты для специфических пользовательских задач

\input{Diploma/setup}      % Упрощённые настройки шаблона

\input{common/newnames}         % Новые переменные, для всего проекта

\input{common/data}             % Основные сведения
\input{common/fonts}            % Определение шрифтов (частичное)
\input{common/styles}           % Стили общие для диссертации и автореферата
\input{Diploma/disstyles}  % Стили для диссертации
\input{Diploma/userstyles} % Стили для специфических пользовательских задач

%%% Библиография. Выбор движка для реализации %%%
\ifnumequal{\value{bibliosel}}{0}{%
    \input{biblio/predefined}   % Встроенная реализация с загрузкой файла через движок bibtex8
}{
    \input{biblio/biblatex}     % Реализация пакетом biblatex через движок biber
}

%%% Управление компиляцией отдельных частей диссертации %%%
% Необходимо сначала иметь полностью скомпилированный документ, чтобы все
% промежуточные файлы были в наличии
% Затем, для вывода отдельных частей можно воспользоваться командой \includeonly
% Ниже примеры использования команды:
%
%\includeonly{Dissertation/part2}
%\includeonly{Dissertation/contents,Dissertation/appendix,Dissertation/conclusion}
%
% Если все команды закомментированы, то документ будет выведен в PDF файл полностью

\begin{document}

\input{common/renames}                 % Переопределение именований

%%% Структура диссертации (ГОСТ Р 7.0.11-2011, 4)
% Титульный лист (ГОСТ Р 7.0.11-2001, 5.1)
\thispagestyle{empty}
\begin{center}
\thesisOrganization
\end{center}

%
\vspace{0pt plus1fill} %число перед fill = кратность относительно некоторого расстояния fill, кусками которого заполнены пустые места
\begin{center}
{\thesisFacultyTitle} \\
{\thesisCafedrTitle} \\
{\thesisProfileTitle}
\end{center}
%
\vspace{0pt plus3fill} %число перед fill = кратность относительно некоторого расстояния fill, кусками которого заполнены пустые места
\begin{center}
\textbf { %\MakeUppercase
\thesisTitle}

\vspace{0pt plus2fill} %число перед fill = кратность относительно некоторого расстояния fill, кусками которого заполнены пустые места
{%\small
Бакалаврская работа \\
Направление \thesisSpecialtyNumber \space
\thesisSpecialtyTitle
}

\vspace{0pt plus2fill} %число перед fill = кратность относительно некоторого расстояния fill, кусками которого заполнены пустые места
\end{center}
%
\vspace{0pt plus6fill} %число перед fill = кратность относительно некоторого расстояния fill, кусками которого заполнены пустые места
\begin{flushright}
	\begin{tabular}{lclr}
		Зав.~кафедрой 	& \underline{\hspace{3cm}} 	& \genCafedraRegaliaShort 	& \genCafedraFIOShort 	\\
		Обучающийся 	& \underline{\hspace{3cm}} 	&   						& \thesisAuthorShort 	\\
		Руководитель 	& \underline{\hspace{3cm}} 	& \supervisorRegaliaShort	& \supervisorFioShort 	\\
	\end{tabular}
	
	%	Зав.~кафедрой\space\underline{\hspace{3cm}}\space\genCafedraRegaliaShort\space\genCafedraFIOShort \\
	%	Обучающийся\space\underline{\hspace{3cm}}\hspace{3.2cm}\thesisAuthorShort \\
	%	Руководитель\space\underline{\hspace{3cm}}\space\supervisorRegaliaShort\space\supervisorFioShort 		
\end{flushright}
%
\vspace{0pt plus2fill} %число перед fill = кратность относительно некоторого расстояния fill, кусками которого заполнены пустые места
{\centering\thesisCity,~\thesisYear\par}
           % Титульный лист
\include{Diploma/contents}        % Оглавление
\chapter*{Введение}                         % Заголовок
\addcontentsline{toc}{chapter}{Введение}    % Добавляем его в оглавление

%\newcommand{\actuality}{}
%\newcommand{\progress}{}
%\newcommand{\aim}{{\textbf\aimTXT}}
%\newcommand{\tasks}{\textbf{\tasksTXT}}
%\newcommand{\novelty}{\textbf{\noveltyTXT}}
%\newcommand{\influence}{\textbf{\influenceTXT}}
%\newcommand{\methods}{\textbf{\methodsTXT}}
%\newcommand{\defpositions}{\textbf{\defpositionsTXT}}
%\newcommand{\reliability}{\textbf{\reliabilityTXT}}
%\newcommand{\probation}{\textbf{\probationTXT}}
%\newcommand{\contribution}{\textbf{\contributionTXT}}
%\newcommand{\publications}{\textbf{\publicationsTXT}}


Современная разработка програмного обеспечения стремительно движется в~сторону более открытого и~гибкого взаимодействия с~пользователем системы (программист, аналитик и~т.д.). 

Одним из~способов взаимодействия с~пользователем является графический интерфейс. Такой подход позволяет скрыть детали работы приложения и~предоставить возможность взаимодействовать с~системой только с~помошью графических элементов управления.

Другим инструментом взаимодействия пользователя с~приложением являются предметно-ориентированные языки программирования (англ. Domain Specific Language, DSL). Основным отличием языка~DSL от~графического интерфейса является гибкость, которую DSL предоставляет пользователю приложения. С~помощью языка~DSL можно описать сценарии взаимодействия с~системой. 

Однако, у~такого подхода существуют и~недостатки. Например, для того, чтобы~пользователь начал писать код на~DSL, ему требуется время на~изучение синтаксиса языка. Одним из решений является выбор среды разработки, которая позволяла бы~писать код на~языке DSL так~же~удобно, как и~на~языках общего применения. 

В данной работе рассматривается язык GRIP~DSL для предметной области класса DMP (Data Management Platform) c~возможностью аналитики в~реальном времени, разработкой которой занимается компания Gridfore. Язык GRIP~DSL предназначен для разработки сценариев управления потоками данных, с~которым могли бы~взаимодействовать пользователи, не~имеющие опыта разработки на~языках программирования общего применения.

Целью работы является подсветка синтаксиса и автодополнение языка программирования GRIP~DSL в~интегрированной среде разработки IntelliJ~IDEA.

Задачи работы: 
\begin{enumerate} 
	\item{Изучить существующие подходы к~лексическому и~синтаксическому анализу языков программирования с~контекстно-свободной грамматикой и~динамической компиляцией.}	
	\item{Разработать лексический и синтаксический анализаторы с~целью подсветки синтаксиса и~автодополнения исходного кода для языка GRIP~DSL.}
\end{enumerate} 

Областью исследования является лексический и~синтаксический анализ, а~также интеграция этих процессов с~современной средой разработки.

Предметом исследования является подсветка и автодополнение синтаксиса Groovy~DSL в~Intellij~IDEA. 

Работа не~включает в~себя разработку полноценного плагина для IDE, самостоятельного компилятора и~предметно-ориентированного языка.


%\textbf{Объем и структура работы.}
%% на случай ошибок оставляю исходный кусок на месте, закомментированным
%Полный объём диссертации составляет  \ref*{TotPages}~страницу
%с~\totalfigures{}~рисунками и~\totaltables{}~таблицами. Список литературы
%содержит \total{citenum}~наименований.
%
%Полный объём дипломной работы составляет
%\formbytotal{TotPages}{страниц}{у}{ы}{}, включая
%\formbytotal{totalcount@figure}{рисун}{ок}{ка}{ков} и
%\formbytotal{totalcount@table}{таблиц}{у}{ы}{}.   Список литературы содержит
%\formbytotal{citenum}{наименован}{ие}{ия}{ий}.
    % Введение
\chapter{Аналитическая часть} \label{chapt1}

\section{Основные понятия в теории компиляции} \label{sec11} 

\subsection{Лексический анализ} \label{sub111}

Основная задача лексического анализатора состоит в чтении входных символов исходной программы, их группировании в лексемы и вывод последовательностей токенов для всех лексем исходной программы. Поток токенов пересылается синтаксическому анализатору для разбора. Так же лексический анализатор может удалять комментарии, пробельные символы, синхронизировать сообщения об ошибках или раскрывать макросы.

Фаза анализа компиляции разделена на лексический и синтаксический анализ по ряду причин: 

\begin{enumerate} 
	\item{Упрощение разработки. Отделение лексического анализа от синтаксического позволяет упростить как минимум одну из фаз анализа. Например, включить в синтаксический анализатор работу с комментариями и пробельными символами существенно сложнее, чем удалить их лексическим анализатором.}
	\item{Увеличение эффективности компилятора. Отдельный лексический анализатор позволяет применять более специализированные методики, предназначенные исключительно для решения лексических задач.}
	\item{Увеличение переносимости компилятора. Особенности устройств ввода могут быть ограниченны лексическим анализатором.}
\end{enumerate}


При рассмотрении лексического анализа используются три связанных, но различных термина:


\begin{itemize} 
	\item{\textit{Токен} представляет собой пару, состоящую из имени токена и необязательного атрибута. Имя токена~--- абстрактный символ, представляющий тип лексической единицы, например конкретное ключевое слово или последовательность символов, составляющих идентификатор. Имена токенов являются входными символами, обрабатываемыми синтаксическим анализатором.}
	\item{\textit{Шаблон}~--- это описание вида, который может принимать лексема токена. В случае ключевого слова шаблон представляет собой последовательность символов, образующая это ключевое слово. Для некоторых токенов шаблон представляет более сложную структуру.}
	\item{\textit{Лексема}~--- последовательность символов исходной программы, которая соответствует шаблону токена и идентифицируется лексическим анализатором как экзепляр токена.}
	\item{Атрибут токена~--- строка или структура, объединяющая несколько блоков информации, которая содержит значение числа, в случае если токен соответствует шаблону числа, или описание лексемы (строчка кода, значение) которая представляет токен. \todo{Пример?}}
\end{itemize}


\todo{Таблица примеров токена?}


\subsection{Формальное определение контекстно-свободной грамматики} \label{sub114}

Существует множество грамматик (грамматики типа 3, контекстно-свободные грамматики, контекстно-зависимые грамматики и грамматики без ограничений), с помощью которых можно описать формальные языки. Для описания языков программирования используются контекстно-свободные грамматики  (далее КС-грамматики).

КС-грамматика используется для определения синтаксиса языка программирования. КС-грамматика естественным образом описывает иерархическую структуру множества конструкций языка программирования. Состоит она из четырёх частей:

\begin{enumerate} 
	\item{\textit{Терминалы}~--- базовые символы, формирующие строки. Термин <<имя токена>> является синонимом слова <<терминал>>. Пример терминала: $\bnfts{if}$, $\bnfts{else}$, $\bnfts{(}$, $\bnfts{)}$}.
	\item{\textit{Нетерминалы}~--- синтаксические переменные, которые обозначают множества строк. В примере $\bnfpn{statement}$ и $\bnfpn{expression}$ являются нетерминалами. Эти множества строк, обозначаемые нетерминалами, помогают определить язык, порождаемый КС-грамматикой. Нетерминалы также налагают на язык иерархическую структуру, облегчающую синтаксический анализ и трансляцию.}
	\item{\textit{Стартовый символ}~--- один из нетерминалов, который обозначает множество строк, являющиеся языком, определяемым КС-грамматикой. По соглашению, продукции стартового символа указываются первыми.}
	\item{\textit{Продукция}~--- способ, которым терминалы и нетерминалы объединяются в строки. Каждая продукция состоит из заголовка(левая часть), символа $\bnfpo$ и тела (правая часть), которое состоит из нуля или некоторого количества терминалов и нетерминалов.}
\end{enumerate}


\begin{bnf*}
	\bnfprod{expression}
	{\bnfpn{expression} \bnfsp \bnfts{+} \bnfsp \bnfpn{term}}\\
	\bnfprod{expression}
	{\bnfpn{expression} \bnfsp \bnfts{-} \bnfsp \bnfpn{term}}\\
	\bnfprod{expression}
	{\bnfpn{term}}\\
	\bnfprod{term}
	{\bnfpn{term} \bnfsp \bnfts{*} \bnfsp \bnfpn{factor}}\\
	\bnfprod{term}
	{\bnfpn{term} \bnfsp \bnfts{/} \bnfsp \bnfpn{factor}}\\
	\bnfprod{term}
	{\bnfpn{factor}}\\
	\bnfprod{factor}
	{\bnfts{(} \bnfsp \bnfpn{expression} \bnfts{)}}\\
\end{bnf*} 
			  % Глава 1
\chapter{Практическая часть} \label{chapt2}

\section{Постановка задачи} \label{sub21}

Целью работы является обеспечение поддержки (подсветки синтаксиса и~автодополнения исходного кода) предметно-ориентированного языка программирования GRIP~DSL в~интерактивной среде разработки (IDE) IntelliJ~IDEA.

Задачи работы: 

\begin{enumerate} 
\item{Изучить существующие подходы к~лексическому и~синтаксическому анализу языков программирования с~контекстно-свободной грамматикой и~динамической компиляцией.}	
\item{Разработать GDSL-модуль для подсветки синтаксиса \\ и~автодополнения исходного кода для языка GRIP~DSL в~интерактивной среде разработки Intellij~IDEA.}
\end{enumerate} 

Областью исследования является лексический и~синтаксический анализ, а~также интеграция этих процессов с~современной средой разработки.

Предметом исследования является подсветка и автодополнение синтаксиса Groovy~DSL в~Intellij~IDEA. 

Работа не~включает в~себя разработка полноценного плагина для IDE, самостоятельного компилятора и~предметно-ориентированного языка.

\section{Средства реализации} \label{sub22}
\section{Требования к программному и аппаратному обеспечению} \label{sub23}

Требования к аппаратному и программному обеспечению: 

\begin{itemize}
\item{RAM: 1 Гб минимум, 2 Гб рекомендовано;}
\item{свободное место на диске: 300 Мб + не менее 1 Гб для кэша;}
\item{минимальное разрешение экрана — 1024x768;}
\item{JDK 8 и выше;}
\item{Groovy 2.4 и выше;}
\item{Intellij IDEA 9 и выше.}
\end{itemize}

\section{Реализация} \label{sub24}
\section{Интерфейс пользователя} \label{sub25} 

\subsection{Подготовка окружения пользователя} \label{subsub251}

Для того, что бы~IntelliJ~IDEA начала взаимодействовать с~GDSL модулем достаточно поместить модуль в~директорию проекта как показанно на~рис.~\ref{img:user-1}.

\begin{figure}[h!]
	\centering
	\includegraphics [scale=0.75] {user1}
	\caption{Интеграция GDSL модуля c~проектом пользователя}
	\label{img:user-1}
\end{figure}

Сам же~проект, согласно требованиям GRIP DSL, должен соблюдать структуру, которая представлена на~рис.~\ref{img:user-2}.

\begin{figure}[h!]
	\centering
	\includegraphics [scale=0.7] {user2}
	\caption{Структура проекта пользователя}
	\label{img:user-2}
\end{figure}

\subsection{Определение метаданных} \label{subsub252}

\textit{Метаданные}~--- набор выражений на~GRIP~DSL, который определяет имена и~типы полей, с~которыми будет работать пользователь.

В~пакете \texttt{metadata.layer.area} должен находиться один или несколько groovy скриптов, внутри которых определяются поля и~их~типы с~помощью GRIP~DSL, как показано на~рис.~\ref{img:user-3}

\begin{figure}[h!]
	\centering
	\includegraphics [scale=0.5] {user3}
	\caption{Определение метаданных}
	\label{img:user-3}
\end{figure}

\subsection{Поддержа метаданных в~сценариях исполнения} \label{subsub253}

\textit{Сценарий исполнения}~--- groovy скрипт, написанный на~диалекте GRIP~DSL, в~котором пользователь определяет интеграцию со~множеством хранилищ данных, их~обработку и~сохранение (\textit{Extract-Transform-Load, ETL}). Согласно GRIP~DSL, сценарий исполнения должен находиться в~пакете \texttt{scenario}.

В~процессе написания сценария исполнения пользователь использует метаданные, описанные в~пакете \texttt{metadata.layer.area} для более удобного и~гибкого описания ETL процесса. Поддержка метаданных в~сценарии исполнения представлена на~рис.~\ref{img:user-4}.

\begin{figure}[h!]
	\centering
	\includegraphics [scale=0.65] {user4}
	\caption{Подсветка синтаксиса и автодополнение метаданных}
	\label{img:user-4}
\end{figure}
\section{План тестирования} \label{sub26}

\subsection*{Тест~1. Проверка отсутствия подсветки у неопределённых полей}

Цель: проверить отсутствие подсветки полей, которые не~являются членами скрипта метаданных.

Порядок выполнения:

\begin{enumerate} 
\item{Создать пустой скрипт определения метаданных.}
\item{Ввести название любой переменной в~скрипте сценария исполнения, не~определённой в~контексте данного скрипта.}
\item{Убедиться, что редактор сообщил об~ошибке.}
\end{enumerate}

В результате редактор кода должен сообщить о~том, что такой переменной не~существует, как показанно на~рис.~\ref{img:test-1}.

\begin{figure}[h!]
	\centering
	\includegraphics [scale=0.7] {test1}
	\caption{Пример отсутствия подсветки у~поля, не~определённого в~скрипте метаданных}
	\label{img:test-1}
\end{figure}

В~случае не~прохождения теста редактор не~сообщит об~ошибке.

\subsection*{Тест 2. Проверка подсветки полей}

Цель: проверить подсветку поля, которое определенно в~скрипте метаданных.

Порядок выполнения:

\begin{enumerate} 
	\item{Определить поле \texttt{TEST\_FIELD} типа \texttt{String} в~скрипте метаданных.}
	\item{Ввести название переменную \texttt{TEST\_FIELD} в~скрипте сценария исполнения.}
	\item{Убедиться, что редактор подсветил переменную.}
\end{enumerate}

В~результате среда разработки должна определить поле как~уже существующее, как~показанно на~рис.~\ref{img:test-2}.

\begin{figure}[h!]
	\centering
	\includegraphics [scale=0.7] {test2}
	\caption{Пример подсветки поля, определённого в~скрипте метаданных}
	\label{img:test-2}
\end{figure}

В~случае не~прохождения теста редактор не~подсветит переменную.

\subsection*{Тест 3. Проверка автодополнения}

Цель: проверить автодополнение имени поля во~время набора текста.

Порядок выполнения:

\begin{enumerate} 
	\item{Определить поле \texttt{TEST\_FIELD} типа \texttt{String} в~скрипте метаданных.}
	\item{Начать вводить название переменной \texttt{TEST\_FIELD} в~скрипте сценария исполнения.}
	\item{Убедиться, что редактор предлагает авто-дополнение к~имени переменной.}
\end{enumerate}

В результате среда разработки должна предложить автодополнение поля при~написании его имени, как~показанно на~рис.~\ref{img:test-3}.

\begin{figure}[h!]
	\centering
	\includegraphics [scale=0.65] {test3}
	\caption{Пример автодополнения поля, определённого в~скрипте метаданных}
	\label{img:test-3}
\end{figure}

В~случае не~прохождения теста редактор не~предложит автодополнение.

\subsection*{Тест 4. Проверка определения типов}

Цель: убедиться в~корректной проверки типов.

Порядок выполнения:

\begin{enumerate} 
	\item{Определить поле \texttt{TEST\_FIELD} типа \texttt{String} в~скрипте метаданных.}
	\item{В~скрипте сценария исполнения написать выражение \\ 
		\texttt{Integer~i~=~TEST\_FIELD~+~1}.}
	\item{Убедиться, что редактор сообщает о~некорректности данного выражения из-за несоответствия типов.}
\end{enumerate}

В результате среда разработки должна указать на несоответствие типов, как показанно на~рис.~\ref{img:test-4}.

\begin{figure}[h!]
	\centering
	\includegraphics [scale=0.7] {test4}
	\caption{Пример определения типа у~поля, определённого в~скрипте метаданных}
	\label{img:test-4}
\end{figure}

В~случае не~прохождения теста редактор идентифицирует выражение корректным, либо сообщение об~ошибке не~будет касаться типов.
			  % Глава 2
\section{Постановка задачи} \label{sub21}

Целью работы является обеспечение поддержки (подсветки синтаксиса и~автодополнения исходного кода) предметно-ориентированного языка программирования GRIP~DSL в~интерактивной среде разработки (IDE) IntelliJ~IDEA.

Задачи работы: 

\begin{enumerate} 
\item{Изучить существующие подходы к~лексическому и~синтаксическому анализу языков программирования с~контекстно-свободной грамматикой и~динамической компиляцией.}	
\item{Разработать GDSL-модуль для подсветки синтаксиса \\ и~автодополнения исходного кода для языка GRIP~DSL в~интерактивной среде разработки Intellij~IDEA.}
\end{enumerate} 

Областью исследования является лексический и~синтаксический анализ, а~также интеграция этих процессов с~современной средой разработки.

Предметом исследования является подсветка и автодополнение синтаксиса Groovy~DSL в~Intellij~IDEA. 

Работа не~включает в~себя разработка полноценного плагина для IDE, самостоятельного компилятора и~предметно-ориентированного языка.
  % Постановка задачи
%\chapter{Анализ задачи} \label{chapt2}

\section{Анализ существующих подходов и инструментов} \label{sect1_1}
\todo{Тут будет обзор всех существующих инструментов и подходов. 
	\cite{Wulf1981,Wimmer2013,Wimmer2012,Vergu2018,VanDeursen2000,Tomita1985,Thatcher1981,Temkin2003,Tanenbaum1983,Takahashi1975,Sujeeth2014,Stadler2013,Spinellis2001,Sikkel1997,Ruchkin2017,Rigger2016a,Rigger2016,Price2018,Pool2016,Paulson1982,Pang2006,Niephaus2018,Moore2000,Mernik2005,Maleki2011,Lovato1995,Louden1997,Levine1992,Leijen1999,Lee1996,Knuth1968,Kitaura2018,Jorring1986,Jones1985,Johnson1975,Javed2004,Homescu2013,Hearnden2002,Grosch1990,Grimmer2014,Griffiths1965,Glanville1978,Gaikwad2018,Ferro1994,Ekman2007,Earley1983,Earley1970,Duboscq2014,Duboscq2013,Deursen1998,Cooper2002,Consel2005,Chugh2016,Chen2018,Brantner2017,Bhamidipaty1998,Barik2018,Aho2003a,Aho2003,Aho1972,Aho1971,Aarts1997}
}

\input{Diploma/data/analyze_Griffiths1965} 
\input{Diploma/data/analyze_Knuth1968} 
\subsection{Эффективный алгоритм разбора контекстно-свободных грамматик} \label{subsection_earley1970}
В~статье~\cite{Earley1970} описывается <<Эффективный алгоритм>> разбора контекстно-свободных грамматик \todo{Джея Йорлей}~и его сравнение~с другими алгоритмами (Кнута, Гриффинса и Патрика).

В разделе~2 приведены основные термины статьи. В~3~и~4 разделах содержится описание алгоритма, 5~раздел описывает эффективность алгоритма. В~6 разделе приведены эмпирические сравнения алгоритма с другими. 7 раздел содержит описание практического применения алгоритма.

\todo{Полезная информация, думаю, стоит включить}  
\subsection{Трансляции контекстно свободных граматик} \label{subsection_Aho1971}
В~вырезке из~книги \cite{Aho1971} содержатся математические основы методов трансляции контекстно-свободных граматик. Приведены теореммы и леммы (с доказательствами) корректности таких методов.

\todo{Стоит рассматривать глубже только в случае необходимости приведения теорем и лемм. Материал сухой и сугубо научный.} 
\subsection{Yacc, статья Стефана Джонсона} \label{subsection_Johnson1975}
Статья <<Еще один компилятор-компилятор>> \cite{Johnson1975} описывает Yacc. Yacc~--- это инструмент, предоставляющий пользователю возможность описать входящие структуры, а затем связать их с некоторыми действиями. По сути, Yacc является мощным инструментом разбора входящего потока строк, который потом предпринимает опредеоленные пользователем действия в случае совпадения некоторых правил, которы так же определены пользователем. 

Статья достаточно полно покрывает основные возможности Yacc, а так же приводит примеры их использования.

\todo{Yacc в любом случаю должен быть подробно разобран, статья хорошая, но пока оставлю ее на первом уровне, так как ожидаю найти более полный материал}           % Анализ задачи
\chapter*{Заключение}                       % Заголовок
\addcontentsline{toc}{chapter}{Заключение}  % Добавляем его в оглавление

%% Согласно ГОСТ Р 7.0.11-2011:
%% 5.3.3 В заключении диссертации излагают итоги выполненного исследования, рекомендации, перспективы дальнейшей разработки темы.
%% 9.2.3 В заключении автореферата диссертации излагают итоги данного исследования, рекомендации и перспективы дальнейшей разработки темы.
%% Поэтому имеет смысл сделать эту часть общей и загрузить из одного файла в автореферат и в диссертацию:

В~результате проделанной работы были изучены существующие подходы к~лексическому и~синтаксическому анализу языков программирования с~контекстно-свободной грамматикой и~динамической компиляцией, проведено исследование существующих подходов к~поддержке DSL в~интегрированной среде разработки IntelliJ IDEA,  а также обеспечена поддержка (подсветки синтаксиса и~авто-дополнения исходного кода) DSL в~IntelliJ~IDEA.




      % Заключение
\include{Diploma/references}      % Список литературы

%%% Настройки для приложений
\appendix
% Оформление заголовков приложений ближе к ГОСТ:
\setlength{\midchapskip}{20pt}
\renewcommand*{\afterchapternum}{\par\nobreak\vskip \midchapskip}
\renewcommand\thechapter{\Asbuk{chapter}} % Чтобы приложения русскими буквами нумеровались


\end{document}
